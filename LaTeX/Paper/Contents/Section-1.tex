\section[Introduction]{Introduction$^{\text{ AA, JE, MK}}$} 
\label{sec:Introduction}

That international trade and specialization can potentially generate welfare gains has long been recognized by economic literature. Especially in developed nations, the service sector has become increasingly important. This development holds true for cross-border services. Since the 1990s trade in services had the highest growth rates compared to other dimensions of international trade (\cite{Breinlich_and}). As global value chains become more integrated with the emergence of cross-border services that can be used as input factors in production (\cite{understanding_2023}), keeping track of the patterns of trade is an ongoing task for empirical research. 

A robust finding is that the gravity model introduced by \textcite{tinbergen1962shaping} can be utilized as an econometric tool to analyze the determinants of trade and identify anomalies that have important policy implications (\cite[p. 41]{krugman2019}). Moreover, the model has been progressively enhanced by empirical refinements, such as the extension by \textcite{Anderson2003} to transfer the originally bilateral model into a multilateral one. \enquote{Hundreds of papers have used the gravity equation to study the effects of geography, demographics, RTAs, tariffs, exports subsidies, embargoes, trade sanctions, the World Trade Organization member- ship, currency unions, foreign aid, immigration, foreign direct investment, cultural ties, trust, reputation, mega sporting events (Olympic Games and World Cup), melting ice caps, etc. on international trade.} (\cite[p. 17]{yotov2016advanced}). This quote summarize the wide applicability of the model quit well.

The original gravity model was however designed for countries that engage in trade of physical merchandise products (from here on we simply call that \textit{trade in goods}), not for cross-border services. Recent literature continuous to adjust the gravity framework, making use of newly developed methods, to also apply the intuitive idea of gravity to other subjects of trade. In this seminar paper, we highlight the differences of gravity based approaches in respect to traded goods and traded services. We aim at providing a comprehensive and useful overview as an introduction to this large literature branch.

Trade barriers and related costs are of particular interest in the empirical literature and in theoretical models. The findings suggest that these play a huge role in determining bilateral trade flows with regards to the establishment of relationships, but also to the volume traded. Therefore, we believe that trade barriers and costs have a high policy relevance. Furthermore, we find that the assessment of trade barriers and costs tends to be a more demanding task for empirical analysis of trade in services, than for trade in goods. This can mainly be explained by the intangible nature of services, the high level of firm heterogeneity in service sectors and the complex interdependencies in global value chains.


The rest of this paper is organized as follows. Section \ref{sec:Gravity_Framework} presents the gravity model by discussing its theoretical linkages to international trade theory and reviewing the empirical tools and challenges that readers have to be aware of when dealing with gravity literature. We then dedicate section \ref{sec:Determinants_of_Trade} to the exploration barriers and costs of trade. Section \ref{sec:Empirical_evidence} discusses implications and results of selected papers for both goods and services applications and finally, section \ref{sec:conclusion} concludes.














































































