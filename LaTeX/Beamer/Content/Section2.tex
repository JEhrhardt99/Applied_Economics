\section{Section 2}






\begin{frame}{Model}


\vspace{-0.7cm}



\begin{block}{Pass-Through Rate}

\begin{equation}
\label{eq:Pass_Through_Rate}
\hat{p} = - \frac{\hat{\beta}}{FTD} 
\end{equation}

\end{block}



Following \textcite{Frondel2024}

\end{frame}



% ========================================================



\begin{frame}{Model}

\vspace{-0.7cm}

\begin{block}{Baseline Model}

\begin{equation}
\begin{aligned}
\label{eq:Base_DiD}
p_{it} &= \beta \cdot FTD_{t} \times GER_{i} &+ \gamma_{i} + \tau_{t} + \epsilon_{it} \\
&= \beta \cdot Treated_{i} &+ \gamma_{i} + \tau_{t} + \epsilon_{it}
\end{aligned}
\end{equation}

Following \textcite{Frondel2024}

\end{block}



\begin{itemize}
    \item FE clustered at the group level
    \item Accounts for errors that are correlated within group over time
    \item would be overconfident (i.e., too small) otherwise
\end{itemize}

\begin{center}
(\cite{Huntington2022})   
\end{center}




\end{frame}



% ========================================================




\begin{frame}{Model 2}




\begin{block}{Differential Pass-Through over Time}

\begin{equation}
\label{eq:Diff_DiD}
p_{it} = \sum_{t=1}^{T} \beta \cdot Day_{t} \times GER_{i} + \gamma_{i} + \tau_{t} + \epsilon_{it}
\end{equation}

Following \textcite{Frondel2024}

\end{block}


    
\end{frame}




% ========================================================





\begin{frame}{Model 2}


We can do that! Difference-in-differences can be modified just a bit to allow the effect to differ in each time period. In other words, we can have dynamic treatment effects. This lets you see things like the effect taking a while to work, or fading out.

While more broadly used, application of this method often refers to \textcite{Autor2001}


    
\end{frame}


% ========================================================

