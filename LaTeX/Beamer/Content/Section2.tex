\section{Methodology}






\begin{frame}{Model}


\vspace{-0.7cm}



\begin{block}{Pass-Through Rate}

\begin{equation}
\label{eq:Pass_Through_Rate}
\hat{p} = - \frac{\hat{\beta}}{FTD} 
\end{equation}

\end{block}



Following \textcite{Frondel2024}

\end{frame}



% ========================================================



\begin{frame}{Baseline Model Intro}

\vspace{-0.7cm}

\begin{block}{Baseline Model}

\begin{equation}
\begin{aligned}
\label{eq:Base_DiD1}
p_{it} &= \beta \cdot FTD_{t} \times GER_{i} &+ \gamma_{i} + \tau_{t} + \epsilon_{it} \\
&= \beta \cdot Treated_{i} &+ \gamma_{i} + \tau_{t} + \epsilon_{it}
\end{aligned}
\end{equation}

\begin{tiny}
    Following \textcite{Frondel2024}
\end{tiny}

\end{block}

\begin{outline}
    \1 $FTD_{t}$: Dummy for treatment period
    \1 $GER_{i}$: Dummy for treatment group
    \1 $\gamma_{i}$: Station fixed effects
    \1 $\tau_{t}$: Time (or day) fixed effects
    \1 $\epsilon$: Idiosyncratic error term
\end{outline}






\end{frame}




% ========================================================





\begin{frame}{Baseline Model}

\vspace{-0.7cm}

\begin{block}{Baseline Model}

\begin{equation}
\begin{aligned}
\label{eq:Base_DiD2}
p_{it} &= \beta \cdot FTD_{t} \times GER_{i} &+ \gamma_{i} + \tau_{t} + \epsilon_{it} \\
&= \beta \cdot Treated_{i} &+ \gamma_{i} + \tau_{t} + \epsilon_{it}
\end{aligned}
\end{equation}

\begin{tiny}
Following \textcite{Frondel2024}    
\end{tiny}

\end{block}



\begin{outline}
    \1 FE clustered at the group level (\textit{treatment} or \textit{control})
    \2 Accounts for errors that are correlated within group over time
    \2 would be overconfident (i.e., too small) otherwise
\end{outline}

\begin{center}
(\cite{Huntington2022})   
\end{center}




\end{frame}






% ========================================================





\begin{frame}{Baseline Model + Competition Metric}

\vspace{-0.9cm}

\begin{block}{Baseline Model}

\small
\vspace{-0.4cm}
\begin{equation}
\begin{aligned}
\label{eq:Base_DiD3}
p_{it} = \beta \cdot FTD_{t} \times GER_{i} + \gamma_{i} + \tau_{t} + \epsilon_{it} 
\end{aligned}
\end{equation}

\begin{tiny}
Following \textcite{Frondel2024}    
\end{tiny}

\end{block}



\begin{block}{Baseline Model + Own Metric}

\small
\vspace{-0.6cm}
\begin{equation}
\begin{aligned}
\label{eq:Base_DiD4}
p_{it} = \beta \cdot FTD_{t} \times GER_{i} + \delta \cdot (Comp_{i} \cdot FTD_{t} \times GER_{i}) + \gamma_{i} + \tau_{t} + \epsilon_{it} 
\end{aligned}
\end{equation}

\vspace{-0.3cm}
\begin{tiny}
    Following own approach
\end{tiny}


\end{block}

\vspace{-0.3cm}

\begin{outline}
\small
    \1 $Comp_{i}$: Nr. of Stations in a 5km radios of station $i$
    \1 $\beta$: \textcolor{BrickRed}{?}Treatment effect for a station with $Comp_{i} = 0$\textcolor{BrickRed}{?}
    \1 $\delta$: \textcolor{BrickRed}{?}The average impact of competition on the treatment effect ($\beta$)\textcolor{BrickRed}{?}
    \1 \textcolor{BrickRed}{?}Play around with log specifications\textcolor{BrickRed}{?}
\end{outline}


\end{frame}







% ========================================================




\begin{frame}{Heterogeneity over Time}


\vspace{-1.2cm}


\begin{block}{Differential Pass-Through over Time}

\begin{equation}
\label{eq:Diff_DiD}
p_{it} = \sum_{t=1}^{T} \beta \cdot Day_{t} \times GER_{i} + \gamma_{i} + \tau_{t} + \epsilon_{it}
\end{equation}

\begin{tiny}
    Following \textcite{Frondel2024}
\end{tiny}


\end{block}

\pause

According to \textcite{Huntington2022}, what is actually estimated is $p_{it} = \sum_{t=-59}^{90} \beta \cdot Day_{t} \times GER_{i} + \gamma_{i} + \tau_{t} + \epsilon_{it}$ with $\beta_{-59}$ … until $\beta_{90}$ and $\beta_{0}$ (Coefficient of the interaction on the day before the treatment - 2022-05-31) \pause

However, $\beta_{-59}$ up to $\beta_{-1}$ are assumed to be insignificant \textbf{and} close to 0 (also a test for parallel trends). 

    
\end{frame}




% ========================================================





\begin{frame}


\enquote{We can do that! Difference-in-differences can be modified just a bit to allow the effect to differ in each time period. In other words, we can have dynamic treatment effects. This lets you see things like the effect taking a while to work, or fading out.} (\cite{Huntington2022}). \newline

\enquote{While more broadly used, application of this method often refers to \textcite{Autor2001}} (\cite{Huntington2022}).


    
\end{frame}














% ========================================================





\begin{frame}{Own Extension}

\vspace{-1.cm}

\begin{block}{Differential Pass-Through over Time with Competition}

\small

\vspace{-0.3cm}

% Resize the equation to fit within the block width
\begin{equation}
\resizebox{\textwidth}{!}{
$\displaystyle
p_{it} = \sum_{t=-59}^{90} (\beta_{t} \cdot Day_{t} \times GER_{i} + \delta_{i} \cdot Day_{t} \times GER_{i} \times Comp_{i}) + \gamma_{i} + \tau_{t} + \epsilon_{it}
$}
\label{eq:Diff_DiD4}
\end{equation}


\vspace{-0.2cm}

\begin{center}
    or
\end{center}

\vspace{-0.2cm}

% Resize the second equation
\begin{equation}
\resizebox{\textwidth}{!}{
$\displaystyle
p_{it} = \sum_{t=-59}^{90} \beta_{t} \cdot Day_{t} \times GER_{i} + \sum_{t=-59}^{90} \delta_{t} \cdot Day_{t} \times GER_{i} \times Comp_{i} + \gamma_{i} + \tau_{t} + \epsilon_{it}
$}
\label{eq:Diff_DiD5}
\end{equation}

\vspace{0.2cm}

\begin{tiny}
    Following own approach
\end{tiny}

\end{block}

\end{frame}





% ========================================================



\begin{frame}{Frame Title}

%     Warum e10 und nicht e5?


% -> Gruppierten Datensatz nach ​


% Für Stationid erzeugen:
%     Anzahl Beobachtungen für E10 insgesamt; dann können Stationen rausfgefiltert werden, die bbspw. weniger als 50/100/200 Beobachtungen haben. Ist das nur ein kleiner Anteil? Dann unkritisch.
    
    
    
    
%     Erstes Datum 2021-12-31 mitnehmen
    
%     df <- df |>
%     tidyr::fill(price, direction = "down")









% -> Gruppierten Datensatz nach ​


% Für Stationid erzeugen:
%     Anzahl Beobachtungen für E10 insgesamt; dann können Stationen rausfgefiltert werden, die bbspw. weniger als 50/100/200 Beobachtungen haben. Ist das nur ein kleiner Anteil? Dann unkritisch.
    
    
    
    
%     Erstes Datum 2021-12-31 mitnehmen
    
%     https://www.rdocumentation.org/packages/zoo/versions/1.8-12/topics/na.locf
    
%     maxgap z.B. auf 7 setzen
    
    
%     Millionen von Beobachtungen im data.frame, die vor Angst schreien und dann verstummen. 

    
    
\end{frame}