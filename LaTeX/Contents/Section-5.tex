\section[Conclusion]{Conclusion$^{\text{ JE, MK}}$}
\label{sec:conclusion}

In this seminar paper, we provided an overview on how the gravity framework, the related theoretical approaches and estimation have evolved. We showed that beside the \textit{usual} determinants - economic mass and distance - other factors representing trade costs and barriers, such as cultural distance and geographical features, may not be neglected while constructing a gravity model. Another main finding of this seminar paper is how the importance of different factors varies when considering at goods, on the one, and services, on the other hand. 

The increasing relevance of services in the composition of trade flows demands for an in-depth analysis of services trade as also reflected by the growing literature in this field yielding valuable information and implications to policy makers. Furthermore, combining the gravity-based approaches with other tools, such as social network analysis, gives more detailed insights to complex interdependencies and global value chains. Another conclusion is that, especially in services trade, sector and firm level heterogeneity should not be neglected and disaggregated analysis should be preferred. We predict that gravity equations will not soon disappear from the landscape of empirical literature on international trade. 






















































